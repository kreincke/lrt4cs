% This file originally comes from 'lrt4cs' [(c) 2020 Karsten Reincke,
% https://www.fodina.de/lrt4cs] that is distributed under the terms
% of CC-BY-3.0-DE (= https://creativecommons.org/licenses/by/3.0/)

\documentclass[
  DIV=calc,
  BCOR=5mm,
  11pt,
  headings=small,
  oneside,
  abstract=true,
  toc=bib,
  ngerman,english]{scrartcl}

%%% (1) general configurations %%%
\usepackage[utf8]{inputenc}
\usepackage{a4}
\usepackage[ngerman,english]{babel}

\usepackage[
  backend=biber,
  style=authortitle-dw,
  sortlocale=auto,
]{biblatex}
\ProvidesFile{biblatex.cfg}


\ExecuteBibliographyOptions{
  %%%%%%%%%%%%%%%%%%%%%%%%%%%%%%%%%%%%%%%%%%%%
  % configurations offered by biblatex itself
  % ------------------------------------------
  maxnames=5,  % Truncate author list after 5 authors ...
  minnames=3,  % ... But display at least 3 authors
  autocite=inline,
  hyperref=true,  % Use hyperref package (should be automatically detected, though)
  backref=true,  % Back references from bibliography page to each encounter
  backrefstyle=two,  % Combine back refs if on two consecutive pages
  isbn=true,  % (Dont) print ISBN, ISSN numbers
  autolang=hyphen,
  % track 'reused' csquotes
  citetracker=constrict, %
  loccittracker=constrict, % discriminate different pages (a.a.O) versues same page (ebda)
  opcittracker=constrict,
  idemtracker=constrict,
  ibidtracker=constrict,
  pagetracker=true,
  %%%%%%%%%%%%%%%%%%%%%%%%%%%%%%%%%%%%%%%%%%%%
  % configurations offered by authortitle-dw
  % ------------------------------------------
  annotation =true,
  namefont=normal,
  firstnamefont=normal,
  idemfont=italic,
  ibidemfont=italic,
  idembib=true, % cluster the books of the same author in bib
  idembibformat=idem, % indicate the same author by ders/dies.
  editorstring=brackets, % parens=(Hrsg.) | brackets=[Hrsg.] | normal = , Hrsg.
  nopublisher=false, % insert publisher into bib data
  editionstring=true, % allow strings in the edition field
  % ZNAME VOL (YEAR) Nr. YOURNALNUMBER
  journalnumber=afteryear, %
  %journumstring=h.
  series=afteryear,
  seriesformat=parens,
  addyear=true, %insert year after titel in 'shorttitle' hints => no year in shorttitle
  firstfull=true, % frist quote complete
  edstringincitations=false, %editor and translator only in the first
  citepages=separate, % vollzitat erst mit seiten, dann 'hier: S. 12'
}
% Put your definitions here.

% refine seriesformat by adding a prefix inside of the parens
\renewcommand*{\seriespunct}{=\addspace}

% by default Biblatex-dw uses Autor1/Autor2 in cites
% these redefinitions overwrite that behaviour by
% duplicating the style used for the bibliography
% (S. p. 35 in the German biblatex-dw handbook )
\renewcommand*{\citemultinamedelim}{\addcomma\space}
\renewcommand*{\citefinalnamedelim}{%
\ifnum\value{liststop}>2 \finalandcomma\fi
\addspace\bibstring{and}\space}%
\renewcommand*{\citerevsdnamedelim}{\addspace}

% biblatex-dw does not print 'ders.' / 'dies.' in row cites of the same book
% Additionally, it does not know the diffrence between the same page of
% of the same work and a different page of the same work quoted before.
%
% as soon as biblatex 3.15 is offered by UBUNTU use \bibncpstring[\mkibid]{ibidem}
% instaed of inserting the German string literally as this hack does:
\renewbibmacro*{cite:ibid}{%
{\ifthenelse{\ifloccit}
  {\printtext[bibhyperref]{ \mkibid{idem} \mkibid{ibid}}%
   \global\booltrue{cbx:loccit}}
  {\printtext[bibhyperref]{ \mkibid{idem} \mkibid{loc.cit., }}}
}
}%
\DefineBibliographyStrings{english}{%
  urlseen = {reviewed at},
}
\DefineBibliographyStrings{german}{%
  urlseen = {heruntergeladen am},
}
\endinput


\addbibresource{../bib/lit.verify.bib}
\addbibresource{../bib/lit.main.bib}

% package for improving the grey value and the line feed handling
\usepackage{microtype}

%language specific quoting signs
\usepackage[
  style=german,
  autostyle=true,
]{csquotes}

% language specific hyphenation
%mycsrfk Hyphenation Include Module text
%
% (c) Karsten Reincke, Frankfurt a.M. 2012, ff.
%
% This file is licensed under the Creative Commons Attribution 3.0 Germany
% License (http://creativecommons.org/licenses/by/3.0/de/):
% For details see teh file LICENSE in the top directory
%


\hyphenation{ my-keds there-fo-re}

\babelhyphenation{ my-keds wor-tmi-tfalsch-entr-ennu-ngen}


%%% (3) layout page configuration %%%

% select the visible parts of a page
% S.31: { plain|empty|headings|myheadings }
\pagestyle{headings}

% select the wished style of page-numbering
% S.32: { arabic,roman,Roman,alph,Alph }
\pagenumbering{arabic}
\setcounter{page}{1}

% select the wished distances using the general setlength order:
% S.34 { baselineskip| parskip | parindent }
% - general no indent for paragraphs
\setlength{\parindent}{0pt}
\setlength{\parskip}{1.2ex plus 0.2ex minus 0.2ex}


%- start(footnote-configuration)

\deffootnote[1.5em]{1.5em}{1.5em}{\textsuperscript{\thefootnotemark)\ }}

%for using label as nameref
\usepackage{nameref}

%integrate nomenclature
% mycsrf  Deutsch Nomenclation Declaration Include Module 
%
% (c) Karsten Reincke, Frankfurt a.M. 2012, ff.
%
% This file is licensed under the Creative Commons Attribution 3.0 Germany
% License (http://creativecommons.org/licenses/by/3.0/de/): 
% For details see teh file LICENSE in the top directory

\usepackage[intoc]{nomencl}
\let\abbr\nomenclature
% Deutsche Überschrift
%\renewcommand{\nomname}{Abbreviations}
\renewcommand{\nomname}{Abkürzungen}

\setlength{\nomlabelwidth}{.20\hsize}
\renewcommand{\nomlabel}[1]{#1 \dotfill}
% reduce the line distance
\setlength{\nomitemsep}{-\parsep}
\makenomenclature


% Hyperlinks
\usepackage{hyperref}
\hypersetup{bookmarks=true,breaklinks=true,colorlinks=true,citecolor=blue,draft=false}

\begin{document}

%% use all entries of the bliography
\nocite{*}

%%-- start(titlepage)
\titlehead{Bib\LaTeX}
\subject{Release 1.0}
\title{Test bibliographischer Angaben\footnote{Dieser Text entstammt \emph{lrt4cs} [\ \copyright\ 2020 Karsten Reincke, \href{https://www.fodina.de/lrt4cs}{https://www.fodina.de/lrt4cs} ], das unter der CC-BY-3.0-DE Lizenz (= \href{https://creativecommons.org/licenses/by/3.0/}{https://creativecommons.org/licenses/by/3.0/} ) veröffentlicht wurde.}}

\maketitle
%%-- end(titlepage)
\begin{abstract}
\noindent \itshape
This texts allows to verify the quality of bibliographic reference data. An entry of \texttt{lit.main.bib} is tested in different contexts:
\end{abstract}


%%%%%%%%%%%%%%%%%%%%%%%%%%%%%%%%%%%%%%%%%%%%%%%%%%%%%%%%%%%%%%%%%%%%%%%%
% Replace KantKdrV1974 by the BibtexKey of the dataset you want to test %
%%%%%%%%%%%%%%%%%%%%%%%%%%%%%%%%%%%%%%%%%%%%%%%%%%%%%%%%%%%%%%%%%%%%%%%%

\section{Context-Sensitive Test}
\begin{itemize}
  \item \enquote{complete data} = Initial quotation.\footcite[cf.][123]{KantKdrV1974}
  \item \enquote{idem ibid.} = same work, same page as before\footcite[cf.][123]{KantKdrV1974}
  \item \enquote{idem loc. cit., p.} = same work as before, different page\footcite[cf.][125f]{KantKdrV1974}
  \item \enquote{idem + rest of all data} = same author, different work\footcite[cf.][321]{KantKdpV1974} (simulation\footcite[cf.][42]{KantKdU1974} )
  \item \enquote{Shorttitel} = first work, first page, different context\footcite[cf.][123]{KantKdrV1974}
\end{itemize}

% This file originally comes from 'lrt4cs' [(c) 2020 Karsten Reincke,
% https://www.fodina.de/lrt4cs] that is distributed under the terms
% of CC-BY-3.0-DE (= https://creativecommons.org/licenses/by/3.0/)


% specific abbreviations
\abbr[utb]{UTB}{Uni-Taschenbuch}
\abbr[stw]{stw}{suhrkamp taschenbuch wissenschaft}

% general abbreviations
\abbr[vgl]{vgl.}{vergleiche}
\abbr[aaO]{a.a.O.}{am angegebenen Ort}
\abbr[ebda]{ebda.}{ebenda}
% \abbr[id]{id.}{idem = latin for 'the same', be it a man, woman or a group\ldots}
% \abbr[ibid]{ibid.}{ibidem = latin for 'at the same place'}
\abbr[ifross]{ifross}{Institut für Rechtsfragen der Freien und Open Source
Software}
% \abbr[lc]{l.c.}{loco citato = latin for 'in the place cited'}
\abbr[wp]{wp.}{webpage = Webdokument ohne innere Seitennummerierung}

% mycsrf English Nomenclation Tokens Include Module 
%
% (c) Karsten Reincke, Frankfurt a.M. 2012, ff.
%
% This file is licensed under the Creative Commons Attribution 3.0 Germany
% License (http://creativecommons.org/licenses/by/3.0/de/): 
% For details see teh file LICENSE in the top directory
%

\abbr[afda]{AfdA}{Anzeiger für deutsches Altertum}
\abbr[zfda]{ZfdA}{Zeitschrift für deutsches Altertum und deutsche Literatur [ISSN: 00442518]}
\abbr[zfaw]{}{Zeitschrift für Allgemeine Wissenschaftstheorie / Journal for General Philosophy of Science [ISSN: 0044-2216]}

\printnomenclature
\printbibliography

\end{document}
